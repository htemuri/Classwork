% Options for packages loaded elsewhere
\PassOptionsToPackage{unicode}{hyperref}
\PassOptionsToPackage{hyphens}{url}
%
\documentclass[
]{article}
\usepackage{amsmath,amssymb}
\usepackage{lmodern}
\usepackage{ifxetex,ifluatex}
\ifnum 0\ifxetex 1\fi\ifluatex 1\fi=0 % if pdftex
  \usepackage[T1]{fontenc}
  \usepackage[utf8]{inputenc}
  \usepackage{textcomp} % provide euro and other symbols
\else % if luatex or xetex
  \usepackage{unicode-math}
  \defaultfontfeatures{Scale=MatchLowercase}
  \defaultfontfeatures[\rmfamily]{Ligatures=TeX,Scale=1}
\fi
% Use upquote if available, for straight quotes in verbatim environments
\IfFileExists{upquote.sty}{\usepackage{upquote}}{}
\IfFileExists{microtype.sty}{% use microtype if available
  \usepackage[]{microtype}
  \UseMicrotypeSet[protrusion]{basicmath} % disable protrusion for tt fonts
}{}
\makeatletter
\@ifundefined{KOMAClassName}{% if non-KOMA class
  \IfFileExists{parskip.sty}{%
    \usepackage{parskip}
  }{% else
    \setlength{\parindent}{0pt}
    \setlength{\parskip}{6pt plus 2pt minus 1pt}}
}{% if KOMA class
  \KOMAoptions{parskip=half}}
\makeatother
\usepackage{xcolor}
\IfFileExists{xurl.sty}{\usepackage{xurl}}{} % add URL line breaks if available
\IfFileExists{bookmark.sty}{\usepackage{bookmark}}{\usepackage{hyperref}}
\hypersetup{
  pdftitle={HW6},
  pdfauthor={Harris Temuri},
  hidelinks,
  pdfcreator={LaTeX via pandoc}}
\urlstyle{same} % disable monospaced font for URLs
\usepackage[margin=1in]{geometry}
\usepackage{color}
\usepackage{fancyvrb}
\newcommand{\VerbBar}{|}
\newcommand{\VERB}{\Verb[commandchars=\\\{\}]}
\DefineVerbatimEnvironment{Highlighting}{Verbatim}{commandchars=\\\{\}}
% Add ',fontsize=\small' for more characters per line
\usepackage{framed}
\definecolor{shadecolor}{RGB}{248,248,248}
\newenvironment{Shaded}{\begin{snugshade}}{\end{snugshade}}
\newcommand{\AlertTok}[1]{\textcolor[rgb]{0.94,0.16,0.16}{#1}}
\newcommand{\AnnotationTok}[1]{\textcolor[rgb]{0.56,0.35,0.01}{\textbf{\textit{#1}}}}
\newcommand{\AttributeTok}[1]{\textcolor[rgb]{0.77,0.63,0.00}{#1}}
\newcommand{\BaseNTok}[1]{\textcolor[rgb]{0.00,0.00,0.81}{#1}}
\newcommand{\BuiltInTok}[1]{#1}
\newcommand{\CharTok}[1]{\textcolor[rgb]{0.31,0.60,0.02}{#1}}
\newcommand{\CommentTok}[1]{\textcolor[rgb]{0.56,0.35,0.01}{\textit{#1}}}
\newcommand{\CommentVarTok}[1]{\textcolor[rgb]{0.56,0.35,0.01}{\textbf{\textit{#1}}}}
\newcommand{\ConstantTok}[1]{\textcolor[rgb]{0.00,0.00,0.00}{#1}}
\newcommand{\ControlFlowTok}[1]{\textcolor[rgb]{0.13,0.29,0.53}{\textbf{#1}}}
\newcommand{\DataTypeTok}[1]{\textcolor[rgb]{0.13,0.29,0.53}{#1}}
\newcommand{\DecValTok}[1]{\textcolor[rgb]{0.00,0.00,0.81}{#1}}
\newcommand{\DocumentationTok}[1]{\textcolor[rgb]{0.56,0.35,0.01}{\textbf{\textit{#1}}}}
\newcommand{\ErrorTok}[1]{\textcolor[rgb]{0.64,0.00,0.00}{\textbf{#1}}}
\newcommand{\ExtensionTok}[1]{#1}
\newcommand{\FloatTok}[1]{\textcolor[rgb]{0.00,0.00,0.81}{#1}}
\newcommand{\FunctionTok}[1]{\textcolor[rgb]{0.00,0.00,0.00}{#1}}
\newcommand{\ImportTok}[1]{#1}
\newcommand{\InformationTok}[1]{\textcolor[rgb]{0.56,0.35,0.01}{\textbf{\textit{#1}}}}
\newcommand{\KeywordTok}[1]{\textcolor[rgb]{0.13,0.29,0.53}{\textbf{#1}}}
\newcommand{\NormalTok}[1]{#1}
\newcommand{\OperatorTok}[1]{\textcolor[rgb]{0.81,0.36,0.00}{\textbf{#1}}}
\newcommand{\OtherTok}[1]{\textcolor[rgb]{0.56,0.35,0.01}{#1}}
\newcommand{\PreprocessorTok}[1]{\textcolor[rgb]{0.56,0.35,0.01}{\textit{#1}}}
\newcommand{\RegionMarkerTok}[1]{#1}
\newcommand{\SpecialCharTok}[1]{\textcolor[rgb]{0.00,0.00,0.00}{#1}}
\newcommand{\SpecialStringTok}[1]{\textcolor[rgb]{0.31,0.60,0.02}{#1}}
\newcommand{\StringTok}[1]{\textcolor[rgb]{0.31,0.60,0.02}{#1}}
\newcommand{\VariableTok}[1]{\textcolor[rgb]{0.00,0.00,0.00}{#1}}
\newcommand{\VerbatimStringTok}[1]{\textcolor[rgb]{0.31,0.60,0.02}{#1}}
\newcommand{\WarningTok}[1]{\textcolor[rgb]{0.56,0.35,0.01}{\textbf{\textit{#1}}}}
\usepackage{graphicx}
\makeatletter
\def\maxwidth{\ifdim\Gin@nat@width>\linewidth\linewidth\else\Gin@nat@width\fi}
\def\maxheight{\ifdim\Gin@nat@height>\textheight\textheight\else\Gin@nat@height\fi}
\makeatother
% Scale images if necessary, so that they will not overflow the page
% margins by default, and it is still possible to overwrite the defaults
% using explicit options in \includegraphics[width, height, ...]{}
\setkeys{Gin}{width=\maxwidth,height=\maxheight,keepaspectratio}
% Set default figure placement to htbp
\makeatletter
\def\fps@figure{htbp}
\makeatother
\setlength{\emergencystretch}{3em} % prevent overfull lines
\providecommand{\tightlist}{%
  \setlength{\itemsep}{0pt}\setlength{\parskip}{0pt}}
\setcounter{secnumdepth}{-\maxdimen} % remove section numbering
\ifluatex
  \usepackage{selnolig}  % disable illegal ligatures
\fi

\title{HW6}
\author{Harris Temuri}
\date{4/5/2021}

\begin{document}
\maketitle

\hypertarget{problem-1}{%
\subsection{Problem 1}\label{problem-1}}

The PimaIndiansDiabetes2 {[}in mlbench package{]} data is a built in R
dataset containing 9 variables and 768 cases. Your task is to use all
the other 8 variables to predict the binary dependent variable
`diabetes' telling us whether the subject is diabetic or not (factor
with 2 levels: neg and pos). You will split the data into 80\% training
and 20\% testing, using seed = 123.

\textbf{\large{Solution}}

\begin{enumerate}
\def\labelenumi{(\alph{enumi})}
\tightlist
\item
  Please split the data into 80\% training and 20\% testing using seed
  =123.
\end{enumerate}

\begin{Shaded}
\begin{Highlighting}[]
\CommentTok{\# Problem 1.1}
\CommentTok{\# Split data into 80\% training and 20\% testing}

\FunctionTok{set.seed}\NormalTok{(}\DecValTok{123}\NormalTok{)}

\NormalTok{training }\OtherTok{\textless{}{-}}\NormalTok{ df}\SpecialCharTok{$}\NormalTok{diabetes }\SpecialCharTok{\%\textgreater{}\%}
  \FunctionTok{createDataPartition}\NormalTok{(}\AttributeTok{p=}\FloatTok{0.8}\NormalTok{, }\AttributeTok{list =} \ConstantTok{FALSE}\NormalTok{)}

\NormalTok{trainData }\OtherTok{\textless{}{-}}\NormalTok{ df[training, ]}
\NormalTok{testData }\OtherTok{\textless{}{-}}\NormalTok{ df[}\SpecialCharTok{{-}}\NormalTok{training, ]}
\end{Highlighting}
\end{Shaded}

\begin{enumerate}
\def\labelenumi{(\alph{enumi})}
\setcounter{enumi}{1}
\tightlist
\item
  Then you shall fit a logistic regression model with all the other 8
  predictors using the training data.
\end{enumerate}

\begin{Shaded}
\begin{Highlighting}[]
\CommentTok{\# Problem 1.2}
\CommentTok{\# Logistic Regression Fit}

\NormalTok{model }\OtherTok{\textless{}{-}} \FunctionTok{glm}\NormalTok{(diabetes }\SpecialCharTok{\textasciitilde{}}\NormalTok{ ., }\AttributeTok{data=}\NormalTok{trainData, }\AttributeTok{family =}\NormalTok{ binomial)}
\FunctionTok{summary}\NormalTok{(model)}
\end{Highlighting}
\end{Shaded}

\begin{verbatim}
## 
## Call:
## glm(formula = diabetes ~ ., family = binomial, data = trainData)
## 
## Deviance Residuals: 
##     Min       1Q   Median       3Q      Max  
## -2.5832  -0.6544  -0.3292   0.6248   2.5968  
## 
## Coefficients:
##               Estimate Std. Error z value Pr(>|z|)    
## (Intercept) -1.053e+01  1.440e+00  -7.317 2.54e-13 ***
## pregnant     1.005e-01  6.127e-02   1.640  0.10092    
## glucose      3.710e-02  6.486e-03   5.719 1.07e-08 ***
## pressure    -3.876e-04  1.383e-02  -0.028  0.97764    
## triceps      1.418e-02  1.998e-02   0.710  0.47800    
## insulin      5.940e-04  1.508e-03   0.394  0.69371    
## mass         7.997e-02  3.180e-02   2.515  0.01190 *  
## pedigree     1.329e+00  4.823e-01   2.756  0.00585 ** 
## age          2.718e-02  2.020e-02   1.346  0.17840    
## ---
## Signif. codes:  0 '***' 0.001 '**' 0.01 '*' 0.05 '.' 0.1 ' ' 1
## 
## (Dispersion parameter for binomial family taken to be 1)
## 
##     Null deviance: 398.80  on 313  degrees of freedom
## Residual deviance: 267.18  on 305  degrees of freedom
## AIC: 285.18
## 
## Number of Fisher Scoring iterations: 5
\end{verbatim}

\begin{enumerate}
\def\labelenumi{(\alph{enumi})}
\setcounter{enumi}{2}
\tightlist
\item
  Please use this fitted model based on the training data to predict the
  response variable `diabetes' (whether the subject is diabetic or not)
  for the testing data. Please generate the confusion matrix, and
  report:
\end{enumerate}

\begin{Shaded}
\begin{Highlighting}[]
\CommentTok{\# Predictions}
\NormalTok{probabilities }\OtherTok{\textless{}{-}}\NormalTok{ model }\SpecialCharTok{\%\textgreater{}\%} \FunctionTok{predict}\NormalTok{(testData, }\AttributeTok{type=}\StringTok{"response"}\NormalTok{)}
\NormalTok{predictedClasses }\OtherTok{\textless{}{-}} \FunctionTok{ifelse}\NormalTok{(probabilities }\SpecialCharTok{\textgreater{}} \FloatTok{0.5}\NormalTok{, }\StringTok{"pos"}\NormalTok{, }\StringTok{"neg"}\NormalTok{)}

\CommentTok{\# Prediction accuracy}
\FunctionTok{mean}\NormalTok{(predictedClasses }\SpecialCharTok{==}\NormalTok{ testData}\SpecialCharTok{$}\NormalTok{diabetes)}
\end{Highlighting}
\end{Shaded}

\begin{verbatim}
## [1] 0.7564103
\end{verbatim}

\begin{Shaded}
\begin{Highlighting}[]
\CommentTok{\# Prediction error}
\FunctionTok{mean}\NormalTok{(predictedClasses }\SpecialCharTok{!=}\NormalTok{ testData}\SpecialCharTok{$}\NormalTok{diabetes)}
\end{Highlighting}
\end{Shaded}

\begin{verbatim}
## [1] 0.2435897
\end{verbatim}

\begin{Shaded}
\begin{Highlighting}[]
\CommentTok{\# Confusion matrix}
\NormalTok{cm }\OtherTok{\textless{}{-}} \FunctionTok{confusionMatrix}\NormalTok{(}\FunctionTok{factor}\NormalTok{(predictedClasses), testData}\SpecialCharTok{$}\NormalTok{diabetes, }\AttributeTok{positive =} \StringTok{"pos"}\NormalTok{)}
\NormalTok{cm}
\end{Highlighting}
\end{Shaded}

\begin{verbatim}
## Confusion Matrix and Statistics
## 
##           Reference
## Prediction neg pos
##        neg  44  11
##        pos   8  15
##                                          
##                Accuracy : 0.7564         
##                  95% CI : (0.646, 0.8465)
##     No Information Rate : 0.6667         
##     P-Value [Acc > NIR] : 0.05651        
##                                          
##                   Kappa : 0.4356         
##                                          
##  Mcnemar's Test P-Value : 0.64636        
##                                          
##             Sensitivity : 0.5769         
##             Specificity : 0.8462         
##          Pos Pred Value : 0.6522         
##          Neg Pred Value : 0.8000         
##              Prevalence : 0.3333         
##          Detection Rate : 0.1923         
##    Detection Prevalence : 0.2949         
##       Balanced Accuracy : 0.7115         
##                                          
##        'Positive' Class : pos            
## 
\end{verbatim}

\begin{enumerate}
\def\labelenumi{(\roman{enumi})}
\tightlist
\item
  The overall accuracy;
\end{enumerate}

\begin{Shaded}
\begin{Highlighting}[]
\NormalTok{cm}\SpecialCharTok{$}\NormalTok{overall[}\DecValTok{1}\NormalTok{]}
\end{Highlighting}
\end{Shaded}

\begin{verbatim}
##  Accuracy 
## 0.7564103
\end{verbatim}

\begin{enumerate}
\def\labelenumi{(\roman{enumi})}
\setcounter{enumi}{1}
\tightlist
\item
  The sensitivity (that is, the probability a subject is predicted to be
  diabetic given that he/she was in fact diabetic);
\end{enumerate}

\begin{Shaded}
\begin{Highlighting}[]
\NormalTok{cm}\SpecialCharTok{$}\NormalTok{byClass[}\DecValTok{1}\NormalTok{]}
\end{Highlighting}
\end{Shaded}

\begin{verbatim}
## Sensitivity 
##   0.5769231
\end{verbatim}

\begin{enumerate}
\def\labelenumi{(\roman{enumi})}
\setcounter{enumi}{2}
\tightlist
\item
  The specificity (that is, the probability a subject is predicted to be
  not diabetic given that he/she was in fact not diabetic).
\end{enumerate}

\begin{Shaded}
\begin{Highlighting}[]
\NormalTok{cm}\SpecialCharTok{$}\NormalTok{byClass[}\DecValTok{2}\NormalTok{]}
\end{Highlighting}
\end{Shaded}

\begin{verbatim}
## Specificity 
##   0.8461538
\end{verbatim}

\end{document}
