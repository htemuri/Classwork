% Options for packages loaded elsewhere
\PassOptionsToPackage{unicode}{hyperref}
\PassOptionsToPackage{hyphens}{url}
%
\documentclass[
]{article}
\usepackage{amsmath,amssymb}
\usepackage{lmodern}
\usepackage{ifxetex,ifluatex}
\ifnum 0\ifxetex 1\fi\ifluatex 1\fi=0 % if pdftex
  \usepackage[T1]{fontenc}
  \usepackage[utf8]{inputenc}
  \usepackage{textcomp} % provide euro and other symbols
\else % if luatex or xetex
  \usepackage{unicode-math}
  \defaultfontfeatures{Scale=MatchLowercase}
  \defaultfontfeatures[\rmfamily]{Ligatures=TeX,Scale=1}
\fi
% Use upquote if available, for straight quotes in verbatim environments
\IfFileExists{upquote.sty}{\usepackage{upquote}}{}
\IfFileExists{microtype.sty}{% use microtype if available
  \usepackage[]{microtype}
  \UseMicrotypeSet[protrusion]{basicmath} % disable protrusion for tt fonts
}{}
\makeatletter
\@ifundefined{KOMAClassName}{% if non-KOMA class
  \IfFileExists{parskip.sty}{%
    \usepackage{parskip}
  }{% else
    \setlength{\parindent}{0pt}
    \setlength{\parskip}{6pt plus 2pt minus 1pt}}
}{% if KOMA class
  \KOMAoptions{parskip=half}}
\makeatother
\usepackage{xcolor}
\IfFileExists{xurl.sty}{\usepackage{xurl}}{} % add URL line breaks if available
\IfFileExists{bookmark.sty}{\usepackage{bookmark}}{\usepackage{hyperref}}
\hypersetup{
  pdftitle={Quiz 4},
  pdfauthor={Harris Temuri},
  hidelinks,
  pdfcreator={LaTeX via pandoc}}
\urlstyle{same} % disable monospaced font for URLs
\usepackage[margin=1in]{geometry}
\usepackage{color}
\usepackage{fancyvrb}
\newcommand{\VerbBar}{|}
\newcommand{\VERB}{\Verb[commandchars=\\\{\}]}
\DefineVerbatimEnvironment{Highlighting}{Verbatim}{commandchars=\\\{\}}
% Add ',fontsize=\small' for more characters per line
\usepackage{framed}
\definecolor{shadecolor}{RGB}{248,248,248}
\newenvironment{Shaded}{\begin{snugshade}}{\end{snugshade}}
\newcommand{\AlertTok}[1]{\textcolor[rgb]{0.94,0.16,0.16}{#1}}
\newcommand{\AnnotationTok}[1]{\textcolor[rgb]{0.56,0.35,0.01}{\textbf{\textit{#1}}}}
\newcommand{\AttributeTok}[1]{\textcolor[rgb]{0.77,0.63,0.00}{#1}}
\newcommand{\BaseNTok}[1]{\textcolor[rgb]{0.00,0.00,0.81}{#1}}
\newcommand{\BuiltInTok}[1]{#1}
\newcommand{\CharTok}[1]{\textcolor[rgb]{0.31,0.60,0.02}{#1}}
\newcommand{\CommentTok}[1]{\textcolor[rgb]{0.56,0.35,0.01}{\textit{#1}}}
\newcommand{\CommentVarTok}[1]{\textcolor[rgb]{0.56,0.35,0.01}{\textbf{\textit{#1}}}}
\newcommand{\ConstantTok}[1]{\textcolor[rgb]{0.00,0.00,0.00}{#1}}
\newcommand{\ControlFlowTok}[1]{\textcolor[rgb]{0.13,0.29,0.53}{\textbf{#1}}}
\newcommand{\DataTypeTok}[1]{\textcolor[rgb]{0.13,0.29,0.53}{#1}}
\newcommand{\DecValTok}[1]{\textcolor[rgb]{0.00,0.00,0.81}{#1}}
\newcommand{\DocumentationTok}[1]{\textcolor[rgb]{0.56,0.35,0.01}{\textbf{\textit{#1}}}}
\newcommand{\ErrorTok}[1]{\textcolor[rgb]{0.64,0.00,0.00}{\textbf{#1}}}
\newcommand{\ExtensionTok}[1]{#1}
\newcommand{\FloatTok}[1]{\textcolor[rgb]{0.00,0.00,0.81}{#1}}
\newcommand{\FunctionTok}[1]{\textcolor[rgb]{0.00,0.00,0.00}{#1}}
\newcommand{\ImportTok}[1]{#1}
\newcommand{\InformationTok}[1]{\textcolor[rgb]{0.56,0.35,0.01}{\textbf{\textit{#1}}}}
\newcommand{\KeywordTok}[1]{\textcolor[rgb]{0.13,0.29,0.53}{\textbf{#1}}}
\newcommand{\NormalTok}[1]{#1}
\newcommand{\OperatorTok}[1]{\textcolor[rgb]{0.81,0.36,0.00}{\textbf{#1}}}
\newcommand{\OtherTok}[1]{\textcolor[rgb]{0.56,0.35,0.01}{#1}}
\newcommand{\PreprocessorTok}[1]{\textcolor[rgb]{0.56,0.35,0.01}{\textit{#1}}}
\newcommand{\RegionMarkerTok}[1]{#1}
\newcommand{\SpecialCharTok}[1]{\textcolor[rgb]{0.00,0.00,0.00}{#1}}
\newcommand{\SpecialStringTok}[1]{\textcolor[rgb]{0.31,0.60,0.02}{#1}}
\newcommand{\StringTok}[1]{\textcolor[rgb]{0.31,0.60,0.02}{#1}}
\newcommand{\VariableTok}[1]{\textcolor[rgb]{0.00,0.00,0.00}{#1}}
\newcommand{\VerbatimStringTok}[1]{\textcolor[rgb]{0.31,0.60,0.02}{#1}}
\newcommand{\WarningTok}[1]{\textcolor[rgb]{0.56,0.35,0.01}{\textbf{\textit{#1}}}}
\usepackage{graphicx}
\makeatletter
\def\maxwidth{\ifdim\Gin@nat@width>\linewidth\linewidth\else\Gin@nat@width\fi}
\def\maxheight{\ifdim\Gin@nat@height>\textheight\textheight\else\Gin@nat@height\fi}
\makeatother
% Scale images if necessary, so that they will not overflow the page
% margins by default, and it is still possible to overwrite the defaults
% using explicit options in \includegraphics[width, height, ...]{}
\setkeys{Gin}{width=\maxwidth,height=\maxheight,keepaspectratio}
% Set default figure placement to htbp
\makeatletter
\def\fps@figure{htbp}
\makeatother
\setlength{\emergencystretch}{3em} % prevent overfull lines
\providecommand{\tightlist}{%
  \setlength{\itemsep}{0pt}\setlength{\parskip}{0pt}}
\setcounter{secnumdepth}{-\maxdimen} % remove section numbering
\ifluatex
  \usepackage{selnolig}  % disable illegal ligatures
\fi

\title{Quiz 4}
\author{Harris Temuri}
\date{4/8/2021}

\begin{document}
\maketitle

\hypertarget{problem-1}{%
\subsection{Problem 1}\label{problem-1}}

The banknote.csv data (see attached) were extracted from images that
were taken from genuine and forged banknote-like specimens. Yes, this is
a Catch Me if You Can story. For digitization, an industrial camera
usually used for print inspection was used. The final images have 400x
400 pixels. Wavelet Transform tool were used to extract features from
images. There are 1372 banknotes, and 5 variables:

\textbf{\large{Solution}}

\begin{enumerate}
\def\labelenumi{(\alph{enumi})}
\tightlist
\item
  Please split the data into 80\% training and 20\% testing using seed
  =123.
\end{enumerate}

\begin{Shaded}
\begin{Highlighting}[]
\CommentTok{\# Problem 1.1}
\CommentTok{\# Split data into 80\% training and 20\% testing}

\FunctionTok{set.seed}\NormalTok{(}\DecValTok{123}\NormalTok{)}

\NormalTok{training }\OtherTok{\textless{}{-}}\NormalTok{ df}\SpecialCharTok{$}\NormalTok{class }\SpecialCharTok{\%\textgreater{}\%}
  \FunctionTok{createDataPartition}\NormalTok{(}\AttributeTok{p=}\FloatTok{0.8}\NormalTok{, }\AttributeTok{list =} \ConstantTok{FALSE}\NormalTok{)}

\NormalTok{trainData }\OtherTok{\textless{}{-}}\NormalTok{ df[training, ]}
\NormalTok{testData }\OtherTok{\textless{}{-}}\NormalTok{ df[}\SpecialCharTok{{-}}\NormalTok{training, ]}
\end{Highlighting}
\end{Shaded}

\begin{enumerate}
\def\labelenumi{(\alph{enumi})}
\setcounter{enumi}{1}
\tightlist
\item
  Then you shall fit a logistic regression model with all the other 8
  predictors using the training data.
\end{enumerate}

\begin{Shaded}
\begin{Highlighting}[]
\CommentTok{\# Problem 1.2}
\CommentTok{\# Logistic Regression Fit}

\NormalTok{model }\OtherTok{\textless{}{-}} \FunctionTok{glm}\NormalTok{(class }\SpecialCharTok{\textasciitilde{}}\NormalTok{ ., }\AttributeTok{data=}\NormalTok{trainData, }\AttributeTok{family =}\NormalTok{ binomial)}
\end{Highlighting}
\end{Shaded}

\begin{verbatim}
## Warning: glm.fit: fitted probabilities numerically 0 or 1 occurred
\end{verbatim}

\begin{Shaded}
\begin{Highlighting}[]
\FunctionTok{summary}\NormalTok{(model)}
\end{Highlighting}
\end{Shaded}

\begin{verbatim}
## 
## Call:
## glm(formula = class ~ ., family = binomial, data = trainData)
## 
## Deviance Residuals: 
##      Min        1Q    Median        3Q       Max  
## -2.49913  -0.00005   0.00000   0.00000   1.44236  
## 
## Coefficients:
##             Estimate Std. Error z value Pr(>|z|)    
## (Intercept)  -8.5895     2.1863  -3.929 8.54e-05 ***
## variance      9.3611     2.4703   3.789 0.000151 ***
## skewness      4.6968     1.2673   3.706 0.000210 ***
## curtosis      6.1372     1.6414   3.739 0.000185 ***
## entropy       0.5193     0.4208   1.234 0.217156    
## ---
## Signif. codes:  0 '***' 0.001 '**' 0.01 '*' 0.05 '.' 0.1 ' ' 1
## 
## (Dispersion parameter for binomial family taken to be 1)
## 
##     Null deviance: 1508.568  on 1097  degrees of freedom
## Residual deviance:   33.991  on 1093  degrees of freedom
## AIC: 43.991
## 
## Number of Fisher Scoring iterations: 13
\end{verbatim}

\begin{enumerate}
\def\labelenumi{(\alph{enumi})}
\setcounter{enumi}{2}
\tightlist
\item
  Please use this fitted model based on the training data to predict the
  response variable `diabetes' (whether the subject is diabetic or not)
  for the testing data. Please generate the confusion matrix, and
  report:
\end{enumerate}

\begin{Shaded}
\begin{Highlighting}[]
\CommentTok{\# Predictions}
\NormalTok{probabilities }\OtherTok{\textless{}{-}}\NormalTok{ model }\SpecialCharTok{\%\textgreater{}\%} \FunctionTok{predict}\NormalTok{(testData, }\AttributeTok{type=}\StringTok{"response"}\NormalTok{)}
\NormalTok{predictedClasses }\OtherTok{\textless{}{-}} \FunctionTok{ifelse}\NormalTok{(probabilities }\SpecialCharTok{\textgreater{}} \FloatTok{0.5}\NormalTok{, }\StringTok{"1"}\NormalTok{, }\StringTok{"0"}\NormalTok{)}

\CommentTok{\# Prediction accuracy}
\FunctionTok{mean}\NormalTok{(predictedClasses }\SpecialCharTok{==}\NormalTok{ testData}\SpecialCharTok{$}\NormalTok{class)}
\end{Highlighting}
\end{Shaded}

\begin{verbatim}
## [1] 0.9817518
\end{verbatim}

\begin{Shaded}
\begin{Highlighting}[]
\CommentTok{\# Prediction error}
\FunctionTok{mean}\NormalTok{(predictedClasses }\SpecialCharTok{!=}\NormalTok{ testData}\SpecialCharTok{$}\NormalTok{class)}
\end{Highlighting}
\end{Shaded}

\begin{verbatim}
## [1] 0.01824818
\end{verbatim}

\begin{Shaded}
\begin{Highlighting}[]
\CommentTok{\# Confusion matrix}
\NormalTok{cm }\OtherTok{\textless{}{-}} \FunctionTok{confusionMatrix}\NormalTok{(}\FunctionTok{factor}\NormalTok{(predictedClasses), }\FunctionTok{factor}\NormalTok{(testData}\SpecialCharTok{$}\NormalTok{class), }\AttributeTok{positive =} \StringTok{"1"}\NormalTok{)}
\NormalTok{cm}
\end{Highlighting}
\end{Shaded}

\begin{verbatim}
## Confusion Matrix and Statistics
## 
##           Reference
## Prediction   0   1
##          0 119   2
##          1   3 150
##                                          
##                Accuracy : 0.9818         
##                  95% CI : (0.9579, 0.994)
##     No Information Rate : 0.5547         
##     P-Value [Acc > NIR] : <2e-16         
##                                          
##                   Kappa : 0.963          
##                                          
##  Mcnemar's Test P-Value : 1              
##                                          
##             Sensitivity : 0.9868         
##             Specificity : 0.9754         
##          Pos Pred Value : 0.9804         
##          Neg Pred Value : 0.9835         
##              Prevalence : 0.5547         
##          Detection Rate : 0.5474         
##    Detection Prevalence : 0.5584         
##       Balanced Accuracy : 0.9811         
##                                          
##        'Positive' Class : 1              
## 
\end{verbatim}

\begin{enumerate}
\def\labelenumi{(\roman{enumi})}
\tightlist
\item
  The overall accuracy;
\end{enumerate}

\begin{Shaded}
\begin{Highlighting}[]
\NormalTok{cm}\SpecialCharTok{$}\NormalTok{overall[}\DecValTok{1}\NormalTok{]}
\end{Highlighting}
\end{Shaded}

\begin{verbatim}
##  Accuracy 
## 0.9817518
\end{verbatim}

\begin{enumerate}
\def\labelenumi{(\roman{enumi})}
\setcounter{enumi}{1}
\tightlist
\item
  The sensitivity (that is, the probability a subject is predicted to be
  diabetic given that he/she was in fact diabetic);
\end{enumerate}

\begin{Shaded}
\begin{Highlighting}[]
\NormalTok{cm}\SpecialCharTok{$}\NormalTok{byClass[}\DecValTok{1}\NormalTok{]}
\end{Highlighting}
\end{Shaded}

\begin{verbatim}
## Sensitivity 
##   0.9868421
\end{verbatim}

\begin{enumerate}
\def\labelenumi{(\roman{enumi})}
\setcounter{enumi}{2}
\tightlist
\item
  The specificity (that is, the probability a subject is predicted to be
  not diabetic given that he/she was in fact not diabetic).
\end{enumerate}

\begin{Shaded}
\begin{Highlighting}[]
\NormalTok{cm}\SpecialCharTok{$}\NormalTok{byClass[}\DecValTok{2}\NormalTok{]}
\end{Highlighting}
\end{Shaded}

\begin{verbatim}
## Specificity 
##   0.9754098
\end{verbatim}

\hypertarget{problem-2}{%
\subsection{Problem 2}\label{problem-2}}

Please find a model that best predicts whether the banknote is forged or
genuine using the stepwise variable selection method and the BIC, based
on the entire dataset. Please only use the original variables and do not
include any other variables such as interactions. Please report the
final model and the associated BIC value.

\textbf{\large{Solution}}

\begin{Shaded}
\begin{Highlighting}[]
\NormalTok{BIC }\OtherTok{\textless{}{-}} \FunctionTok{stepAIC}\NormalTok{(model, }\AttributeTok{k=}\FunctionTok{log}\NormalTok{(}\FunctionTok{nrow}\NormalTok{(df)))}
\end{Highlighting}
\end{Shaded}

\begin{verbatim}
## Start:  AIC=70.11
## class ~ variance + skewness + curtosis + entropy
\end{verbatim}

\begin{verbatim}
## Warning: glm.fit: fitted probabilities numerically 0 or 1 occurred
\end{verbatim}

\begin{verbatim}
##            Df Deviance    AIC
## - entropy   1    35.59  64.49
## <none>           33.99  70.11
## - skewness  1   481.03 509.93
## - curtosis  1   591.79 620.69
## - variance  1   906.39 935.28
\end{verbatim}

\begin{verbatim}
## Warning: glm.fit: fitted probabilities numerically 0 or 1 occurred
\end{verbatim}

\begin{verbatim}
## 
## Step:  AIC=64.49
## class ~ variance + skewness + curtosis
## 
##            Df Deviance     AIC
## <none>           35.59   64.49
## - curtosis  1   594.64  616.31
## - skewness  1   642.83  664.51
## - variance  1  1115.29 1136.96
\end{verbatim}

\begin{Shaded}
\begin{Highlighting}[]
\NormalTok{BIC}\SpecialCharTok{$}\NormalTok{anova}
\end{Highlighting}
\end{Shaded}

\begin{verbatim}
## Stepwise Model Path 
## Analysis of Deviance Table
## 
## Initial Model:
## class ~ variance + skewness + curtosis + entropy
## 
## Final Model:
## class ~ variance + skewness + curtosis
## 
## 
##        Step Df Deviance Resid. Df Resid. Dev      AIC
## 1                            1093   33.99056 70.11068
## 2 - entropy  1 1.600442      1094   35.59100 64.48710
\end{verbatim}

Final Model with BIC = 64.49 is

\[
class = (-6.97) + 6.75(variance) + 3.50(skewness) + 4.44(curtosis)
\]

\hypertarget{problem-3}{%
\subsection{Problem 3}\label{problem-3}}

Please find a model that best predicts whether the banknote is forged or
genuine using the best subset variable selection method and the BIC,
based on the entire dataset. Please only use the original variables and
do not include any other variables such as interactions. Please report
the final model and the associated BIC value.

\textbf{\large{Solution}}

\begin{Shaded}
\begin{Highlighting}[]
\NormalTok{dummy }\OtherTok{\textless{}{-}} \FunctionTok{data.frame}\NormalTok{(df)}
\NormalTok{bestSubset }\OtherTok{\textless{}{-}}\NormalTok{ bestglm}\SpecialCharTok{::}\FunctionTok{bestglm}\NormalTok{(dummy, }\AttributeTok{IC=}\StringTok{"BIC"}\NormalTok{, }\AttributeTok{family=}\NormalTok{binomial)}
\end{Highlighting}
\end{Shaded}

\begin{verbatim}
## Morgan-Tatar search since family is non-gaussian.
\end{verbatim}

\begin{verbatim}
## Warning: glm.fit: fitted probabilities numerically 0 or 1 occurred

## Warning: glm.fit: fitted probabilities numerically 0 or 1 occurred
\end{verbatim}

\begin{Shaded}
\begin{Highlighting}[]
\NormalTok{bestSubset}
\end{Highlighting}
\end{Shaded}

\begin{verbatim}
## BIC
## BICq equivalent for q in (0, 0.870796809815784)
## Best Model:
##              Estimate Std. Error   z value     Pr(>|z|)
## (Intercept) -6.884973  1.3838479 -4.975238 6.516756e-07
## variance     6.783457  1.3949643  4.862818 1.157263e-06
## skewness     3.506680  0.6932163  5.058564 4.224245e-07
## curtosis     4.464192  0.9006030  4.956892 7.162970e-07
\end{verbatim}

\begin{Shaded}
\begin{Highlighting}[]
\NormalTok{bestSubset}\SpecialCharTok{$}\NormalTok{BestModel}
\end{Highlighting}
\end{Shaded}

\begin{verbatim}
## 
## Call:  glm(formula = y ~ ., family = family, data = Xi, weights = weights)
## 
## Coefficients:
## (Intercept)     variance     skewness     curtosis  
##      -6.885        6.783        3.507        4.464  
## 
## Degrees of Freedom: 1371 Total (i.e. Null);  1368 Residual
## Null Deviance:       1885 
## Residual Deviance: 53.3  AIC: 61.3
\end{verbatim}

Final Model with BIC = 61.3 is

\[
class = (-6.885) + 6.783(variance) + 3.507(skewness) + 4.464(curtosis)
\]

\end{document}
