\documentclass{article}

\usepackage{fancyhdr}
\usepackage{extramarks}
\usepackage{amsmath}
\usepackage{amsthm}
\usepackage{amsfonts}
\usepackage{tikz}
\usepackage[plain]{algorithm}
\usepackage{algpseudocode}
\usepackage{enumitem}
\usepackage{listings}
\usepackage{color}
\usepackage{textgreek}
\usepackage{multirow,array}

\definecolor{dkgreen}{rgb}{0,0.6,0}
\definecolor{gray}{rgb}{0.5,0.5,0.5}
\definecolor{mauve}{rgb}{0.58,0,0.82}

\lstset{ %
  language=R,                     % the language of the code
  basicstyle=\footnotesize,       % the size of the fonts that are used for the code
  numbers=left,                   % where to put the line-numbers
  numberstyle=\tiny\color{gray},  % the style that is used for the line-numbers
  stepnumber=1,                   % the step between two line-numbers. If it's 1, each line
                                  % will be numbered
  numbersep=5pt,                  % how far the line-numbers are from the code
  backgroundcolor=\color{white},  % choose the background color. You must add \usepackage{color}
  showspaces=false,               % show spaces adding particular underscores
  showstringspaces=false,         % underline spaces within strings
  showtabs=false,                 % show tabs within strings adding particular underscores
  frame=single,                   % adds a frame around the code
  rulecolor=\color{black},        % if not set, the frame-color may be changed on line-breaks within not-black text (e.g. commens (green here))
  tabsize=2,                      % sets default tabsize to 2 spaces
  captionpos=b,                   % sets the caption-position to bottom
  breaklines=true,                % sets automatic line breaking
  breakatwhitespace=false,        % sets if automatic breaks should only happen at whitespace
  title=\lstname,                 % show the filename of files included with \lstinputlisting;
                                  % also try caption instead of title
  keywordstyle=\color{blue},      % keyword style
  commentstyle=\color{dkgreen},   % comment style
  stringstyle=\color{mauve},      % string literal style
  escapeinside={\%*}{*)},         % if you want to add a comment within your code
  morekeywords={*,...}            % if you want to add more keywords to the set
}


\usetikzlibrary{automata,positioning}

%
% Basic Document Settings
%

\topmargin=-0.45in
\evensidemargin=0in
\oddsidemargin=0in
\textwidth=6.5in
\textheight=9.0in
\headsep=0.25in

\linespread{1.1}

\pagestyle{fancy}
\lhead{\hmwkAuthorName}
\chead{\hmwkClass\ (\hmwkClassInstructor): \hmwkTitle}
\rhead{\firstxmark}
\lfoot{\lastxmark}
\cfoot{\thepage}

\renewcommand\headrulewidth{0.4pt}
\renewcommand\footrulewidth{0.4pt}

\setlength\parindent{0pt}

%
% Create Problem Sections
%

\newcommand{\enterProblemHeader}[1]{
    \nobreak\extramarks{}{Problem \arabic{#1} continued on next page\ldots}\nobreak{}
    \nobreak\extramarks{Problem \arabic{#1} (continued)}{Problem \arabic{#1} continued on next page\ldots}\nobreak{}
}

\newcommand{\exitProblemHeader}[1]{
    \nobreak\extramarks{Problem \arabic{#1} (continued)}{Problem \arabic{#1} continued on next page\ldots}\nobreak{}
    \stepcounter{#1}
    \nobreak\extramarks{Problem \arabic{#1}}{}\nobreak{}
}

\setcounter{secnumdepth}{0}
\newcounter{partCounter}
\newcounter{homeworkProblemCounter}
\setcounter{homeworkProblemCounter}{1}
\nobreak\extramarks{Problem \arabic{homeworkProblemCounter}}{}\nobreak{}

%
% Homework Problem Environment
%
% This environment takes an optional argument. When given, it will adjust the
% problem counter. This is useful for when the problems given for your
% assignment aren't sequential. See the last 3 problems of this template for an
% example.
%
\newenvironment{homeworkProblem}[1][-1]{
    \ifnum#1>0
        \setcounter{homeworkProblemCounter}{#1}
    \fi
    \section{Problem \arabic{homeworkProblemCounter}}
    \setcounter{partCounter}{1}
    \enterProblemHeader{homeworkProblemCounter}
}{
    \exitProblemHeader{homeworkProblemCounter}
}

%
% Homework Details
%   - Title
%   - Due date
%   - Class
%   - Section/Time
%   - Instructor
%   - Author
%

\newcommand{\hmwkTitle}{Homework 2}
\newcommand{\hmwkDueDate}{09/24}
\newcommand{\hmwkClassTime}{}
\newcommand{\hmwkClass}{AMS 316.01}
\newcommand{\hmwkClassInstructor}{Haipeng Xing}
\newcommand{\hmwkAuthorName}{Harris Temuri}
%
% Title Page
%

\title{
    \vspace{2in}
    \textmd{\textbf{\hmwkClass:\ \hmwkTitle}}\\
    \normalsize\vspace{0.1in}\small{Due\ on\ \hmwkDueDate}\\
    \vspace{0.1in}\large{\textit{\hmwkClassInstructor\ \hmwkClassTime}}
    \vspace{3in}
}

\author{\hmwkAuthorName}
\date{}

\renewcommand{\part}[1]{\textbf{\large Part \Alph{partCounter}}\stepcounter{partCounter}\\}

%
% Various Helper Commands
%

% Useful for algorithms
\newcommand{\alg}[1]{\textsc{\bfseries \footnotesize #1}}

% For derivatives
\newcommand{\deriv}[1]{\frac{\mathrm{d}}{\mathrm{d}x} (#1)}

% For partial derivatives
\newcommand{\pderiv}[2]{\frac{\partial}{\partial #1} (#2)}

% Integral dx
\newcommand{\dx}{\mathrm{d}x}

% Alias for the Solution section header
\newcommand{\solution}{\textbf{\large Solution}}

% Probability commands: Expectation, Variance, Covariance, Bias
\newcommand{\E}{\mathrm{E}}
\newcommand{\Var}{\mathrm{Var}}
\newcommand{\Cov}{\mathrm{Cov}}
\newcommand{\Bias}{\mathrm{Bias}}

\begin{document}

\maketitle

\pagebreak

\begin{homeworkProblem}

    The following data show the coded sales of company X in successive 4-week periods over
    1995-1998.
    
    \begin{table}[h]
        \begin{tabular}{|l|lllllllllllll|}
        \hline
             & I   & II & III & IV & V & VI & VII & VIII & IX & X & XI & XII & XIII \\ \hline
        1995 & 153 & 189& 221 & 215&302& 223&  201& 173  & 121&106&  86&   87& 108  \\
        1996 & 133& 177& 241 &228 &283& 255& 238& 164& 128& 108& 87& 74 &95  \\
        1997 & 145 &200& 187& 201& 292& 220& 233& 172& 119& 81& 65& 76& 74      \\
        1998 & 111 &170& 243 &178 &248 &202 &163 &139& 120 &96& 95 &53& 94  \\ \hline
        \end{tabular}
    \end{table}
    
    \begin{enumerate}[label=\alph*., start = 1]
        \item Plot the data.
        \item Assess the trend and seasonal effects.
    \end{enumerate}
    
    \textbf{\large{Solution}}\\\\
    \textbf{Part A}
    \\\\
    Create data on excel \\
    \\
    \includegraphics[scale = 0.6]{data_png.PNG}
    \\
    \newpage
    Import data to R-Studio
    \begin{lstlisting}
    data_2 <- read_excel("C:/Users/Harris/Desktop/AMS 316/Homework 2/data_2.xlsx")
    data_2$date <- as.Date('1995-01-01')
    data_2$date <- as.Date(data_2$date, '%Y/%m/%d')
    for (i in 1:51) {
     week(data_2$date[i+1]) <- week(data_2$date[i])+4
     year(data_2$date[i+1]) <- year(data_2$date[i])+(data_2$year-1995)
    }
    for (i in 1:51) {
     year(data_2$date[i+1]) <- year(data_2$date[i])+(data_2$year[i+1]-data_2$year[i])
    }
    ggplot(data = data_2, aes(date, sales)) + geom_line()
    \end{lstlisting}
    \begin{figure}[h]
    \caption{Plot of sales over time}
    \centering
    \includegraphics[scale=1]{P1_plot.pdf}
    \end{figure}
    \\
    \newpage
    \textbf{Part B}
    \\
    \begin{lstlisting}
    x <- c(153, 189, 221, 215, 302, 223, 201, 173, 121, 106, 86, 87, 108, 133, 177, 241, 228, 283, 255, 238, 164, 128, 108, 87, 74, 95, 145, 200, 187, 201, 292, 220, 233, 172, 119, 81, 65, 76, 74, 111, 170, 243, 178, 248, 202, 163, 139, 120, 96, 95, 53, 94)
    ts_data <- ts(x, frequency = 13)
    plot(decompose(ts_data))
    \end{lstlisting}
    \begin{figure}[h]
    \caption{Decomposed time series of Company X data}
    \centering
    \includegraphics[scale=1]{P1_decompose.pdf}
    \end{figure}
    
    \\Looking at the graph, we can see a general downtrend and an additive seasonality.

\end{homeworkProblem}
\pagebreak
\begin{homeworkProblem}
    Sixteen successive observations on a stationary time series are as follows: 1.6, 0.8, 1.2, 0.5,0.9, 1.1, 1.1, 0.6, 1.5, 0.8, 0.9, 1.2, 0.5, 1.3, 0.8, 1.2
    \begin{enumerate}[label =\alph*., start = 1]
        \item Plot the data.
        \item Looking at the graph plotted in (a), guess an approximate value for the auto-correlation coefficient at lag 1.
        \item Plot $x_t$ against $X_{t+1}$, and again try to guess the value of $r_1$
        \item Calculate $r_1$
    \end{enumerate}
    
    \textbf{\large{Solution}}\\\\
    \textbf{Part A}
    \\\\
    Create data frame

    \begin{lstlisting}
    df <- c(1.6, 0.8, 1.2, 0.5,0.9, 1.1, 1.1, 0.6,1.5, 0.8, 0.9, 1.2, 0.5, 1.3, 0.8, 1.2)
    plot.ts(df)
    plot(df)
    \end{lstlisting}
    \begin{figure}[h]
    \caption{Line plot of observations over time}
    \centering
    \includegraphics[scale=1]{P2_plot1.pdf}
    \end{figure}
    \newpage
    \begin{lstlisting}
    plot(df)
    \end{lstlisting}
    \begin{figure}[h]
    \caption{Scatter plot of observations over time}
    \centering
    \includegraphics[scale=1]{P2_plot1_scatter.pdf}
    \end{figure}
    
    \textbf{Part B}
    \\\\
    Looking at the graph, I cannot really tell what the value of $r_1$ might be.
    \\
    \newpage
    \textbf{Part C}
    \begin{lstlisting}
    x_t <- df[1:15]
    'x_t+1' <- df[2:16]
    plot(x_t, 'x_t+1')
    abline(lm('x_t+1' ~ 'x_t))
    \end{lstlisting}
    \begin{figure}[h]
    \caption{Scatter plot with trend line of $x_t$ vs $x_{t+1}$}
    \centering
    \includegraphics[scale=1]{P2_partc.pdf}
    \end{figure}
    Looking at this plot, I would say $r_1$ is about -0.6.\\
    \\
    \textbf{Part D}
    \begin{lstlisting}
    acf(df)[]
    
    # Autocorrelations of series 'df', by lag
    # 
    #     0      1      2      3      4      5      6      7      8      9     10 
    # 1.000 -0.549  0.250 -0.104 -0.165  0.067  0.037 -0.116  0.122  0.098 -0.189 
    #     11     12 
    #  0.220 -0.305 
    \end{lstlisting}
    The value of $r_1$ is -0.549.
    
\end{homeworkProblem}
\pagebreak
\begin{homeworkProblem}
    A computer generates a series of 400 observations that are supposed to be random. The first
    10 sample auto-correlation of the series are $r_1 = 0.02$, $r_2 = 0.05$, $r_3 = -0.09$, $r_4 = 0.08$, $r_5 = -0.02$, $r_6 = 0.00$, $r_7 = 0.12$, $r_8 = 0.06$, $r_9 = 0.02$, $r_{10} = -0.08$. Is there any evidence of non-randomness?\\
    
    \textbf{\large{Solution}}\\\\
    According to your slides, 
    \begin{quote}
         ...if a time series is random, we can expect 19 out of 20 of the values of $r_k$ to lie between $\pm1.96/\sqrt{N}$.
    \end{quote}
    \[
        \pm1.96/\sqrt{400} = \pm0.98
    \]
    Looking at the values of the auto-correlation, it seems there is one value which constitutes as significant, $r_7 = 0.12$. However since there is only value out of the 10 which is significant, I don't believe it constitutes as evidence of non-randomness.
\end{homeworkProblem}

\begin{homeworkProblem}
    Suppose we have a seasonal series of monthly observation $\{X_t\}$, for which the seasonal factor
    at time $t$ is denoted by $\{S_t\}$. Further suppose that seasonal pattern is constant through time
    so that $S_t = S_{t-12}$ for all $t$. Denote a stationary series of random deviations by $\{ε_t\}$.
    
    \begin{enumerate}[label = \alph*., start = 1]
        \item  Consider the model $X_t = a + b_t + S_t + ε_t$ having a global linear trend and multiplicative seasonality. Show that the seasonal difference operator $\nabla_{12}$ acts on $X_t$ to produce a stationary series.
        \item  Consider the model $X_t = (a + b_t)S_t + ε_t$ having a global linear trend and multiplicative seasonality. Does the operator $\nabla_{12}$ transform $X_t$ to stationary? If not, define a difference operator that does.
        \\ (Note: As stationary is not formally defined until Chapter 3, you should use heuristic arguments. A stationary process may involve a constant mean value (that could be zero) plus
        any linear combination of the stationary series $\{ε_t\}$, but should not include terms such as
        trend and seasonality.)

    \end{enumerate}
    
    \textbf{\large{Solution}}\\\\
    \textbf{Part A}\\\\
    Since the data has a non varying seasonality, in order to transform the data into a stationary series, we need to remove the trend and seasonality. Using seasonal differencing(first or second order) will remove the both of them for us.
    \begin{gather*}
        X_t = a + b_t + S_t + e_t \\
        \nabla_{12}  = X_t - X_{t-12} \\
        = a + b_t + S_t + e_t - a - b_{t-12} - S_{t-12} - e_{t_12} \\
        = (b_t - b_{t-12}) + (S_t - S_{t-12}) + e_t - e_{t-12} \\
        = c + 0 + e_t - e_{t-12}; c\text{ is a constant}
    \end{gather*}
    The first order seasonal differencing removed both the trend and seasonality from this model.
    \newpage
    \textbf{Part B} \\
    \begin{gather*}
        X_t = (a + b_t)S_t + e_t \\
        \nabla_{12} = x_t - x_{t-12} \\
        = aS_t + b_tS_t + e_t - aS_{t-12} - b_{t-12}S_{t-12} - e_{t-12} \\
        = 0 + (b_tS_t - b_{t-12}S_{t-12}) + (e_t-e_{t-12})\\
        = cS_t + e_t - e_{t-12}; c\text{ is a constant}
    \end{gather*}
    Still has seasonality. Will try second order seasonal differencing.\\
    \begin{gather*}
        X_t = (a + b_t)S_t + e_t \\
        \nabla_{12}^2 = x_t - 2x_{t-12} + x_{t-11}\\
        = (a + b_t)S_t + e_t - 2((a + b_{t-12})S_{t-12} + e_{t_12}) + (a + b_{t-11})S_{t-11} + e_{t-11}\\
        = (a_S_t - 2aS_t + aS_t) + (b_tS_t - 2b_{t-12}S_t + b_{t-11}S_t) + (e_t - 2e_{t-12} + e_{t-11}) \\
        = 0 + (b_t - 2b_{t-12} + b_{t-11})S_t + (e_t - 2e_{t-12} + e_{t-11}) \\
        = (b_t - b_t)S_t + (e_t - 2e_{t-12} + e_{t-11}) \\
        =e_t - 2e_{t-12} + e_{t-11}
    \end{gather*}
    \\
    The second order seasonal differencing removed both the trend and seasonality from this model.

\end{homeworkProblem}

\end{document}
