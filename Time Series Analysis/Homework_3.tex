\documentclass{article}

\usepackage{fancyhdr}
\usepackage{extramarks}
\usepackage{amsmath}
\usepackage{amsthm}
\usepackage{amsfonts}
\usepackage{tikz}
\usepackage[plain]{algorithm}
\usepackage{algpseudocode}
\usepackage{enumitem}
\usepackage{listings}
\usepackage{color}
\usepackage{textgreek}
\usepackage{multirow,array}
\usepackage{nicefrac}
\usepackage{mathtools}% http://ctan.org/pkg/mathtools

\definecolor{dkgreen}{rgb}{0,0.6,0}
\definecolor{gray}{rgb}{0.5,0.5,0.5}
\definecolor{mauve}{rgb}{0.58,0,0.82}

\lstset{ %
  language=R,                     % the language of the code
  basicstyle=\footnotesize,       % the size of the fonts that are used for the code
  numbers=left,                   % where to put the line-numbers
  numberstyle=\tiny\color{gray},  % the style that is used for the line-numbers
  stepnumber=1,                   % the step between two line-numbers. If it's 1, each line
                                  % will be numbered
  numbersep=5pt,                  % how far the line-numbers are from the code
  backgroundcolor=\color{white},  % choose the background color. You must add \usepackage{color}
  showspaces=false,               % show spaces adding particular underscores
  showstringspaces=false,         % underline spaces within strings
  showtabs=false,                 % show tabs within strings adding particular underscores
  frame=single,                   % adds a frame around the code
  rulecolor=\color{black},        % if not set, the frame-color may be changed on line-breaks within not-black text (e.g. commens (green here))
  tabsize=2,                      % sets default tabsize to 2 spaces
  captionpos=b,                   % sets the caption-position to bottom
  breaklines=true,                % sets automatic line breaking
  breakatwhitespace=false,        % sets if automatic breaks should only happen at whitespace
  title=\lstname,                 % show the filename of files included with \lstinputlisting;
                                  % also try caption instead of title
  keywordstyle=\color{blue},      % keyword style
  commentstyle=\color{dkgreen},   % comment style
  stringstyle=\color{mauve},      % string literal style
  escapeinside={\%*}{*)},         % if you want to add a comment within your code
  morekeywords={*,...}            % if you want to add more keywords to the set
}


\usetikzlibrary{automata,positioning}

%
% Basic Document Settings
%

\topmargin=-0.45in
\evensidemargin=0in
\oddsidemargin=0in
\textwidth=6.5in
\textheight=9.0in
\headsep=0.25in

\linespread{1.1}

\pagestyle{fancy}
\lhead{\hmwkAuthorName}
\chead{\hmwkClass\ (\hmwkClassInstructor): \hmwkTitle}
\rhead{\firstxmark}
\lfoot{\lastxmark}
\cfoot{\thepage}

\renewcommand\headrulewidth{0.4pt}
\renewcommand\footrulewidth{0.4pt}

\setlength\parindent{0pt}

%
% Create Problem Sections
%

\newcommand{\enterProblemHeader}[1]{
    \nobreak\extramarks{}{Problem \arabic{#1} continued on next page\ldots}\nobreak{}
    \nobreak\extramarks{Problem \arabic{#1} (continued)}{Problem \arabic{#1} continued on next page\ldots}\nobreak{}
}

\newcommand{\exitProblemHeader}[1]{
    \nobreak\extramarks{Problem \arabic{#1} (continued)}{Problem \arabic{#1} continued on next page\ldots}\nobreak{}
    \stepcounter{#1}
    \nobreak\extramarks{Problem \arabic{#1}}{}\nobreak{}
}

\setcounter{secnumdepth}{0}
\newcounter{partCounter}
\newcounter{homeworkProblemCounter}
\setcounter{homeworkProblemCounter}{1}
\nobreak\extramarks{Problem \arabic{homeworkProblemCounter}}{}\nobreak{}

%
% Homework Problem Environment
%
% This environment takes an optional argument. When given, it will adjust the
% problem counter. This is useful for when the problems given for your
% assignment aren't sequential. See the last 3 problems of this template for an
% example.
%
\newenvironment{homeworkProblem}[1][-1]{
    \ifnum#1>0
        \setcounter{homeworkProblemCounter}{#1}
    \fi
    \section{Problem \arabic{homeworkProblemCounter}}
    \setcounter{partCounter}{1}
    \enterProblemHeader{homeworkProblemCounter}
}{
    \exitProblemHeader{homeworkProblemCounter}
}

%
% Homework Details
%   - Title
%   - Due date
%   - Class
%   - Section/Time
%   - Instructor
%   - Author
%

\newcommand{\hmwkTitle}{Homework 3}
\newcommand{\hmwkDueDate}{10/07}
\newcommand{\hmwkClassTime}{}
\newcommand{\hmwkClass}{AMS 316.01}
\newcommand{\hmwkClassInstructor}{Haipeng Xing}
\newcommand{\hmwkAuthorName}{Harris Temuri}
%
% Title Page
%

\title{
    \vspace{2in}
    \textmd{\textbf{\hmwkClass:\ \hmwkTitle}}\\
    \normalsize\vspace{0.1in}\small{Due\ on\ \hmwkDueDate}\\
    \vspace{0.1in}\large{\textit{\hmwkClassInstructor\ \hmwkClassTime}}
    \vspace{3in}
}

\author{\hmwkAuthorName}
\date{}

\renewcommand{\part}[1]{\textbf{\large Part \Alph{partCounter}}\stepcounter{partCounter}\\}

%
% Various Helper Commands
%

% Useful for algorithms
\newcommand{\alg}[1]{\textsc{\bfseries \footnotesize #1}}

% For derivatives
\newcommand{\deriv}[1]{\frac{\mathrm{d}}{\mathrm{d}x} (#1)}

% For partial derivatives
\newcommand{\pderiv}[2]{\frac{\partial}{\partial #1} (#2)}

% Integral dx
\newcommand{\dx}{\mathrm{d}x}

% Alias for the Solution section header
\newcommand{\solution}{\textbf{\large Solution}}

% Probability commands: Expectation, Variance, Covariance, Bias
\newcommand{\E}{\mathrm{E}}
\newcommand{\Var}{\mathrm{Var}}
\newcommand{\Cov}{\mathrm{Cov}}
\newcommand{\Bias}{\mathrm{Bias}}

\begin{document}

\maketitle

\pagebreak
In the following equations, ${Z_t}$ is a discrete-time, purely random process such that $E(Z_t) = 0, Var(Z_t) = σ^2_Z$, 
and successive values of $Z_t$ are independent so that $Cov(Z_t, Z_{t+k}) = 0, k\neq0$.

\begin{homeworkProblem}

    Show that the ac.f of the second-order MA process
    
    \[
        X_t = Z_t + 0.7Z_{t-1} - 0.2Z_{t-2}
    \]
    
    is given by
    
    \[  p(k) = 
        \begin{cases} 
          1, & k = 0 \\
          0.37, & k = \pm 1 \\
          -0.13, & k = \pm 2\\
          0, & otherwise
        \end{cases}
    \]
    
    \textbf{\large{Solution}}\\\\
    \\\\
    
    \begin{gather*}
        p(k) = 
        \begin{cases} 
          1, & k = 0 \\
          \nicefrac{\sum_{i=0}^{q-k} \beta_i\beta_{i+k}}{\sum_{i=0}^{q} \beta_i^2}, & k = 1,...,q \\
          0, & k \geq q
        \end{cases}
    \end{gather*}
    for k=1,
    \begin{align*}
        &= \sum_{i=0}^{q-1} \beta_i\beta_{i+k}
        \\&= (1*0.7) + (0.7*(-.2))
        \\&= 0.56
    \end{align*}
    \begin{align*}
        &= \sum_{i=0}^{q} \beta_i^2
        \\&= 1^2 + 0.7^2 + (-0.2)^2
        \\&= 1.53
    \end{align*}
    \begin{align*}
        &= \nicefrac{\sum_{i=0}^{q-1} \beta_i\beta_{i+k}}{\sum_{i=0}^{q} \beta_i^2}
        \\&= \nicefrac{0.56}{1.53}
        \\&= 0.366 \cong \textbf{0.37}
    \end{align*}
    \newpage
    for k=2,
    \begin{align*}
        &= \sum_{i=0}^{q-2} \beta_i\beta_{i+k}
        \\&= (1*(-0.2))
        \\&= -0.2
    \end{align*}
    \begin{align*}
        &= \nicefrac{\sum_{i=0}^{q-2} \beta_i\beta_{i+k}}{\sum_{i=0}^{q} \beta_i^2}
        \\&= \nicefrac{-0.2}{1.53}
        \\&= -0.1307 \cong \textbf{-0.13}
    \end{align*}
    
    \[  p(k) = 
        \begin{cases} 
          1, & k = 0 \\
          0.37, & k = \pm 1 \\
          -0.13, & k = \pm 2\\
          0, & otherwise
        \end{cases}
    \]
    
\end{homeworkProblem}

\begin{homeworkProblem}
    Consider the MA($m$) process, with equal weights $\frac{1}{m+1}$ at all lags (so it is a real moving average), given by,
    \[
        X_t = \sum_{k=0}^{m} \frac{Z_{t-k}}{m+1}
    \]
    Show that the ac.f of this process is
    \[
        p(k) = 
        \begin{cases} 
          \frac{m+1-k}{m+1}, & k = 0,...,m \\
          0, & k \geq m
        \end{cases}
    \]
    \textbf{\large{Solution}}\\\\
        \begin{gather*}
        p(k) = 
        \begin{cases} 
          1, & k = 0 \\
          \nicefrac{\sum_{i=0}^{q-k} \beta_i\beta_{i+k}}{\sum_{i=0}^{q} \beta_i^2}, & k = 1,...,m \\
          0, & k \geq q
        \end{cases}
    \end{gather*}
    for $k=0,...,m$,
    % sum of bi and b(i+k)
    \begin{align*}
        &= \sum_{i=0}^{m} \beta_i\beta_{i+k}
        \\&= \frac{m+1-k}{(m+1)^2}
    \end{align*}
    \begin{align*}
        &= \sum_{i=0}^{m} \beta_i^2
        \\&= \frac{1}{m+1}
    \end{align*}
    \begin{align*}
        &= \nicefrac{\sum_{i=0}^{q-1} \beta_i\beta_{i+k}}{\sum_{i=0}^{q} \beta_i^2}
        \\&= \frac{m+1-k}{m+1}
    \end{align*}
    \newpage

    \[  p(k) = 
        \begin{cases} 
          1, & k = 0 \\
          \frac{m+1-k}{m+1}, & k = 1,...,m \\
          0, & otherwise
        \end{cases}
    \]
    
\end{homeworkProblem}
\begin{homeworkProblem}
    Consider the infinite-order MA process $X_t$, defined by
    \[
        X_t = Z_t + C(Z_{t-1}+Z_{t-2}+...)
    \]
    where $C$ is a constant. Show that the process is non-stationary. Also show that the series of the first differences $Y_t$ defined by
    \[
        Y_t = X_t - X_{t-1}
    \]
    is a first-order MA process and is stationary. Find the ac.f of $Y_t$.\\
    
    \textbf{\large{Solution}}\\\\
    $X_t$ is not stationary because its variance changes with time. $Y_t$ is stationary because it forms a purely random process, which is stationary.
    
    \begin{gather*}
        X_t = Z_t + C(Z_{t-1} + Z_{t-2} + ...)
        Y_t = X_t - X_{t-1}
    \end{gather*}
    
    \begin{align*}
        Y_t &= Z_t + C(Z_{t-1} + Z_{t-2} + ...) - (Z_{t-1} + C(Z_{t-2} + Z_{t-3} + ...))
        \\&= Z_t + CZ_{t-1} - Z_{t-1}
        \\&= Z_t + (C-1)Z_{t-1}
    \end{align*}
    
    \begin{gather*}
        p(k) = 
        \begin{cases} 
          1, & k = 0 \\
          \nicefrac{\sum_{i=0}^{q-k} \beta_i\beta_{i+k}}{\sum_{i=0}^{q} \beta_i^2}, & k = 1,...,q \\
          0, & k \geq q
        \end{cases}
    \end{gather*}
    
    for $k=1$,
    
    \begin{align*}
        &= \sum_{i=0}^{q-1} \beta_i\beta_{i+k}
        \\&= C-1
    \end{align*}
    
    \begin{align*}
    &= \sum_{i=0}^{q} \beta_i^2
    \\&= 1+(C-1)^2
    \end{align*}
    
    \begin{align*}
    &= \nicefrac{\sum_{i=0}^{q-1} \beta_i\beta_{i+k}}{\sum_{i=0}^{q} \beta_i^2}
    \\&= \frac{C-1}{1+(C-1)^2}
    \end{align*}
    
    
    \[  p(k) = 
        \begin{cases} 
          1, & k = 0 \\
          \frac{C-1}{1+(C-1)^2}, & k = \pm 1 \\
          0, & otherwise
        \end{cases}
    \]
\end{homeworkProblem}
\end{document}
