\documentclass{article}

\usepackage{fancyhdr}
\usepackage{extramarks}
\usepackage{amsmath}
\usepackage{amsthm}
\usepackage{amsfonts}
\usepackage{tikz}
\usepackage[plain]{algorithm}
\usepackage{algpseudocode}
\usepackage{enumitem}
\usepackage{listings}
\usepackage{color}
\usepackage{textgreek}
\usepackage{multirow,array}

\definecolor{dkgreen}{rgb}{0,0.6,0}
\definecolor{gray}{rgb}{0.5,0.5,0.5}
\definecolor{mauve}{rgb}{0.58,0,0.82}

\lstset{ %
  language=R,                     % the language of the code
  basicstyle=\footnotesize,       % the size of the fonts that are used for the code
  numbers=left,                   % where to put the line-numbers
  numberstyle=\tiny\color{gray},  % the style that is used for the line-numbers
  stepnumber=1,                   % the step between two line-numbers. If it's 1, each line
                                  % will be numbered
  numbersep=5pt,                  % how far the line-numbers are from the code
  backgroundcolor=\color{white},  % choose the background color. You must add \usepackage{color}
  showspaces=false,               % show spaces adding particular underscores
  showstringspaces=false,         % underline spaces within strings
  showtabs=false,                 % show tabs within strings adding particular underscores
  frame=single,                   % adds a frame around the code
  rulecolor=\color{black},        % if not set, the frame-color may be changed on line-breaks within not-black text (e.g. commens (green here))
  tabsize=2,                      % sets default tabsize to 2 spaces
  captionpos=b,                   % sets the caption-position to bottom
  breaklines=true,                % sets automatic line breaking
  breakatwhitespace=false,        % sets if automatic breaks should only happen at whitespace
  title=\lstname,                 % show the filename of files included with \lstinputlisting;
                                  % also try caption instead of title
  keywordstyle=\color{blue},      % keyword style
  commentstyle=\color{dkgreen},   % comment style
  stringstyle=\color{mauve},      % string literal style
  escapeinside={\%*}{*)},         % if you want to add a comment within your code
  morekeywords={*,...}            % if you want to add more keywords to the set
}


\usetikzlibrary{automata,positioning}

%
% Basic Document Settings
%

\topmargin=-0.45in
\evensidemargin=0in
\oddsidemargin=0in
\textwidth=6.5in
\textheight=9.0in
\headsep=0.25in

\linespread{1.1}

\pagestyle{fancy}
\lhead{\hmwkAuthorName}
\chead{\hmwkClass\ (\hmwkClassInstructor\ \hmwkClassTime): \hmwkTitle}
\rhead{\firstxmark}
\lfoot{\lastxmark}
\cfoot{\thepage}

\renewcommand\headrulewidth{0.4pt}
\renewcommand\footrulewidth{0.4pt}

\setlength\parindent{0pt}

%
% Create Problem Sections
%

\newcommand{\enterProblemHeader}[1]{
    \nobreak\extramarks{}{Problem \arabic{#1} continued on next page\ldots}\nobreak{}
    \nobreak\extramarks{Problem \arabic{#1} (continued)}{Problem \arabic{#1} continued on next page\ldots}\nobreak{}
}

\newcommand{\exitProblemHeader}[1]{
    \nobreak\extramarks{Problem \arabic{#1} (continued)}{Problem \arabic{#1} continued on next page\ldots}\nobreak{}
    \stepcounter{#1}
    \nobreak\extramarks{Problem \arabic{#1}}{}\nobreak{}
}

\setcounter{secnumdepth}{0}
\newcounter{partCounter}
\newcounter{homeworkProblemCounter}
\setcounter{homeworkProblemCounter}{1}
\nobreak\extramarks{Problem \arabic{homeworkProblemCounter}}{}\nobreak{}

%
% Homework Problem Environment
%
% This environment takes an optional argument. When given, it will adjust the
% problem counter. This is useful for when the problems given for your
% assignment aren't sequential. See the last 3 problems of this template for an
% example.
%
\newenvironment{homeworkProblem}[1][-1]{
    \ifnum#1>0
        \setcounter{homeworkProblemCounter}{#1}
    \fi
    \section{Problem \arabic{homeworkProblemCounter}}
    \setcounter{partCounter}{1}
    \enterProblemHeader{homeworkProblemCounter}
}{
    \exitProblemHeader{homeworkProblemCounter}
}

%
% Homework Details
%   - Title
%   - Due date
%   - Class
%   - Section/Time
%   - Instructor
%   - Author
%

\newcommand{\hmwkTitle}{Homework 3}
\newcommand{\hmwkDueDate}{10/09}
\newcommand{\hmwkClassTime}{}
\newcommand{\hmwkClass}{ECO 321.03}
\newcommand{\hmwkClassInstructor}{Maria Perez-Urdiales}
\newcommand{\hmwkAuthorName}{Harris Temuri}
%
% Title Page
%

\title{
    \vspace{2in}
    \textmd{\textbf{\hmwkClass:\ \hmwkTitle}}\\
    \normalsize\vspace{0.1in}\small{Due\ on\ \hmwkDueDate}\\
    \vspace{0.1in}\large{\textit{\hmwkClassInstructor\ \hmwkClassTime}}
    \vspace{3in}
}

\author{\hmwkAuthorName}
\date{}

\renewcommand{\part}[1]{\textbf{\large Part \Alph{partCounter}}\stepcounter{partCounter}\\}

%
% Various Helper Commands
%

% Useful for algorithms
\newcommand{\alg}[1]{\textsc{\bfseries \footnotesize #1}}

% For derivatives
\newcommand{\deriv}[1]{\frac{\mathrm{d}}{\mathrm{d}x} (#1)}

% For partial derivatives
\newcommand{\pderiv}[2]{\frac{\partial}{\partial #1} (#2)}

% Integral dx
\newcommand{\dx}{\mathrm{d}x}

% Alias for the Solution section header
\newcommand{\solution}{\textbf{\large Solution}}

% Probability commands: Expectation, Variance, Covariance, Bias
\newcommand{\E}{\mathrm{E}}
\newcommand{\Var}{\mathrm{Var}}
\newcommand{\Cov}{\mathrm{Cov}}
\newcommand{\Bias}{\mathrm{Bias}}

\begin{document}

\maketitle

\pagebreak

\begin{homeworkProblem}

    Let us consider the following regression model for a sample of $n$ female workers
    \begin{gather}
    \text{\textbf{WorkHours}}_i = \beta_0 + \beta_1\text{\textbf{Child}}_i + u_i,
     \text{    }i = 1,...,n,
    \end{gather}

    where ${Child}_i$ is a binary variable that takes value 1 if the individual has one or more
    child and 0 otherwise; and $WorkHours$ is the individual’s usual hours worked per
    week in past 12 months.
    
    \\Let \textbf{WeeklyPay} = $Y$ and \textbf{Child} = $X$. Also let $n_1$ the number of individuals who
    has a one or more than one child and $n_0$ the number of female worker who does not
    have a child, such that $n_0 + n_1 = n$.
    
    \\Recall that the OLS estimators are given by
    
    \begin{gather*}
        \hat{\beta}_1 = \frac{\sum_{i=1}^n (y_i - \Bar{Y})(x_i - \Bar{X})}{\sum_{i=1}^n (x_i - \Bar{X})^2}
        \\\hat{\beta}_0 = \Bar{Y} - \hat{\beta}_1\Bar{X}.
    \end{gather*}

    \begin{enumerate}[label=\alph*., start = 1]
        \item   Let $y_{i,0}$ an individual $i$’s working hours for which the value of $X$ is 0; and $y_{i,1}$
                an individual $i$’s working hour for which the value of $X$ is 1. Show that 
                \begin{gather*}
                    \sum_{i=1}^n (y_i - \Bar{Y})(x_i - \Bar{X}) = \frac{n_0}{n}\sum_{i=1}^{n_0} y_{i,1} -
                    \frac{n_1}{n}\sum_{i=1}^{n_0} y_{i,0}
                \end{gather*}
                
        \item   Show that
                \begin{gather*}
                    \sum_{i=1}^n (x_i - \Bar{X})^2 = \frac{n_1n_0}{n}
                \end{gather*}

        \item   Use the results in (\textit{a}) and (\textit{b}) to conclude that
        \begin{gather*}
            \hat{\beta}_1 = \Bar{Y}_1 - \Bar{Y}_0
        \end{gather*}
                upon an appropriate definition of $\Bar{Y}_1$ and $\Bar{Y}_0$.
        \item   Show that $\Bar{Y} = (n_1/n)\Bar{Y}_1 + (n_0/n)\Bar{Y}_0$
        \item   Using the result in (\textit{d}), finally show that $\hat{\beta}_0 = \Bar{Y}_0$\\
    \end{enumerate}
    
    
    \textbf{\large{Solution}}\\\\
    \textbf{Part A}
    \begin{gather*}
        Y_i = \beta_0  + \beta_1X_i +u_i
        \\y_{i,0} = \beta_0 + u_i
        \\y_{i,1} = \beta_0 + \beta_1 + u_i
        \\\beta_1 = E(Y_i|X_i=1) - E(Y_i|X_i=0)
        \\\beta_1 = \frac{\sum_{i=1}^{n_1} y_{i,1}}{n_1} - \frac{\sum_{i=1}^{n_0} y_{i,0}}{n_0}
        \\\frac{\sum_{i=1}^n (y_i - \Bar{Y})(x_i - \Bar{X})}{\sum_{i=1}^n (x_i - \Bar{X})^2}
        = \frac{\sum_{i=1}^{n_1} y_{i,1}}{n_1} - \frac{\sum_{i=1}^{n_0} y_{i,0}}{n_0}
        \\\frac{\sum_{i=1}^n (y_i - \Bar{Y})(x_i - \Bar{X})}{\frac{n_1n_0}{n}}
        = \frac{\sum_{i=1}^{n_1} y_{i,1}}{n_1} - \frac{\sum_{i=1}^{n_0} y_{i,0}}{n_0}
        \\\sum_{i=1}^n (y_i - \Bar{Y})(x_i - \Bar{X}) = \frac{n_0}{n}\sum_{i=1}^{n_0} y_{i,1} -
        \frac{n_1}{n}\sum_{i=1}^{n_0} y_{i,0}
    \end{gather*}
    \textbf{Part B}\\
    \begin{gather*}
        \\\sum_{i=1}^n (y_i - \Bar{Y})(x_i - \Bar{X}) = \frac{n_0}{n}\sum_{i=1}^{n_0} y_{i,1} -
        \frac{n_1}{n}\sum_{i=1}^{n_0} y_{i,0}
        \\\sum_{i=1}^n (y_i - \Bar{Y})(x_i - \Bar{X}) = \frac{n_1n_0}{n}(\frac{\sum_{i=1}^{n_1} y_{i,1}}{n_1} - \frac{\sum_{i=1}^{n_0} y_{i,0}}{n_0})
        \\\frac{\sum_{i=1}^n (y_i - \Bar{Y})(x_i - \Bar{X})}{\frac{n_1n_0}{n}}
        =\frac{\sum_{i=1}^n (y_i - \Bar{Y})(x_i - \Bar{X})}{\sum_{i=1}^n (x_i - \Bar{X})^2}
        \\\frac{n_1n_0}{n} = \sum_{i=1}^n (x_i - \Bar{X})^2
    \end{gather*}
    \textbf{Part C}\\
    \begin{gather*}
        \hat{\beta}_1 = \frac{\sum_{i=1}^{n_1} y_{i,1}}{n_1} - \frac{\sum_{i=1}^{n_0} y_{i,0}}{n_0}
        \\\hat{\beta}_1 = \frac{n_1\Bar{Y}_1}{n_1} - \frac{n_0\Bar{Y}_0}{n_0}
        \\\hat{\beta}_1 = \bar{Y}_1 - \Bar{Y}_0
    \end{gather*}
    \textbf{Part D}\\
    \begin{gather*}
        \bar{Y} = \frac{\sum_{i=0}^n y_{i}}{n}
        \\\bar{Y} = \frac{\sum_{i=1}^{n_1} y_{i,1} + \sum_{i=1}^{n_0} y_{i,0}}{n}
        \\\bar{Y} = \frac{n_1\bar{Y}_1 + n_0\bar{Y}_0}{n}
        \\\Bar{Y} = (n_1/n)\Bar{Y}_1 + (n_0/n)\Bar{Y}_0
    \end{gather*}
    \textbf{Part E}\\
    \begin{gather*}
    \hat{\beta}_0 = \Bar{Y} - \hat{\beta}_1\Bar{X}
    \\\hat{\beta}_0 = \Bar{Y}_0 - \hat{\beta}_1\Bar{X}
    \\\hat{\beta}_0 = \Bar{Y}_0 - \hat{\beta}_1(0)
    \\\hat{\beta}_0 = \Bar{Y}_0   
    \end{gather*}

\end{homeworkProblem}
\newpage
\begin{homeworkProblem}

    The data file, \textbf{female work.csv}, is a subset of The American Community Survey(ACS) for the year 2018, with only female workers from age 18-40. A detailed description of the variables contained in the file is given in the pdf file \textbf{Documentation.pdf}
    available on Blackboard. You can find a sample code for fitting and testing a linear
    regression model in R inside the folders called ”Chapter 2: Linear Regression with
    One Regressor” and ”Chapter 3: Inference in the Linear Model with One Regressor”
    on Blackboard.
    \begin{enumerate}[label=\alph*., start = 1]
        \item   Suppose that all assumptions for OLS are satisfied and estimate the simple
                regression model in equation (1) using heteroskedasticity robust standard errors.
        \item   Report the values for $\hat{\beta}_0$ and $\hat{\beta}_1$.
        \item   What does the sample statistic $\hat{\beta}_0$ capture?
        \item   Do women with child work less? By how much? Explain.
        \item   Is the estimated effect of having child on women's working hours statistically significant?
                Carry out a test at the 1\% level.
        \item   Construct a 95\% confidence interval for the effect of having children on working hours.\\
    \end{enumerate}
    
    \textbf{\large{Solution}}\\\\
    \textbf{Part A}\\
    \begin{lstlisting}
    model <- lm(workhour ~ child, data = FemaleWork)
    coeftest(model, vcov = vcovHC(model, type = "HC1"))
    # t test of coefficients:
    #
    #              Estimate Std. Error  t value  Pr(>|t|)    
    # (Intercept) 36.250973   0.028318 1280.143 < 2.2e-16 ***
    # child       (neg)1.009660   0.041918  -24.087 < 2.2e-16 ***
    # ---
    # Signif. codes:  0 ‘***’ 0.001 ‘**’ 0.01 ‘*’ 0.05 ‘.’ 0.1 ‘ ’ 1
    \end{lstlisting}
    \begin{gather*}
    \text{\textbf{WorkHours}}_i = \beta_0 + \beta_1\text{\textbf{Child}}_i + u_i,
    \\\text{\textbf{WorkHours}}_i = 36.25 - 1.01\text{\textbf{Child}}_i + u_i,
    \end{gather*}
    
    \textbf{Part B}\\
    The value for $\hat{\beta}_0$ is 36.25 with $SE(\hat{\beta}_0) = 0.028$. The value for $\hat{\beta}_1$ is -1.01 with $SE(\hat{\beta}_1) = 0.042$.\\
    
    \textbf{Part C}\\
    It's an estimate on how many hours a female without a child works.\\
    
    \textbf{Part D}\\
    Women with one or more children tend to work 1.01 fewer hours than women without children.\\
    \newpage
    \textbf{Part E}\\
    \begin{lstlisting}
    summary(model)
    
    # Call:
    # lm(formula = workhour ~ child, data = FemaleWork)
    #
    # Residuals:
    #     Min      1Q  Median      3Q     Max 
    # -35.251  -6.251   3.749   4.759  63.759 
    #
    # Coefficients:
    #             Estimate Std. Error t value Pr(>|t|)    
    # (Intercept) 36.25097    0.02829 1281.50   <2e-16 ***
    # child       -1.00966    0.04193  -24.08   <2e-16 ***
    # ---
    # Signif. codes:  0 ‘***’ 0.001 ‘**’ 0.01 ‘*’ 0.05 ‘.’ 0.1 ‘ ’ 1
    #
    # Residual standard error: 11.96 on 328397 degrees of freedom
    # Multiple R-squared:  0.001763,	Adjusted R-squared:  0.00176 
    # F-statistic: 579.9 on 1 and 328397 DF,  p-value: < 2.2e-16
    \end{lstlisting}
    The p-value for the linear model is much less than 0.01 so the model is statistically significant.\\
    
    \textbf{Part F}\\
    \begin{lstlisting}
    newdata <- data.frame(child=1)
    predict(model, newdata = newdata, interval = "confidence")
    #       fit      lwr      upr
    #  35.24131 35.18066 35.30196
    \end{lstlisting}
    The 95\% confidence interval for the effect of having children on working hours is $[35.18066, 35.30196]$.
    
\end{homeworkProblem}

\end{document}
